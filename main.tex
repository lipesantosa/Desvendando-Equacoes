\documentclass[a4paper, 12pt]{article}
\usepackage[top=3cm, bottom=2cm, left=3cm, right=2cm]{geometry,amsmath,amssymb,graphicx} % Required for inserting images

\begin{document}

\maketitle

\section{Introdução}
Olá!\\

 Esse livro é um projeto que comecei no ano de 2024. Fazia um tempo que eu ja tinha vontade de começar um livro, mas me faltava coragem e sobre o que falar. Então eu tive a ideia de criar algo com foco na matemática básica para aprender física.\\
 
 Aqui serão encontrados conteúdos desde as quatro operações até trigonometria. O foco aqui não é especializar o estudante em matemática, mas sim, aprenseta-lo aos principais conceitos para que ele não tneha dificuldade, seja na escola, universidade ou qualquer outro lugar.\\

 Serão encontrados neste livro toda a parte teórica do assunto, questões resolvidas e outras para praticar. Fica a cargo do leitor a intensidade de estudo e resolução de questões.\\

  É um livro feito por um estudante, para outros estudantes. Deixo claro que no momento em que escrevo este, sou graduando em Licenciatura em Física, ou seja, não é algo feito por um graduado, portador de diploma e da área da Matemática. Mas posso garantir o esforço para tal, e a boa vontade. \\

\newpage
\tableofcontents
\newpage
\section{As Quatro Operações Básicas}
\\
+ - * /\\

 As operações básicas são os processos mais elementares e fundamentais de toda a matemática. Com elas, conseguimos desenvolver desde o mais simples, ao mais díficil. São elas, Adição, Subtração, Multiplicação e Divisão. Cada uma com suas funções e propriedades especificas.\\
 
 Esse será um capitulo rápido. O foco dele é apenas revisar essas operações para as próximas que virão.
\subsection{Adição}
Assim como o próprio nome já é autoexplicativo, a ideia aqui é adicionar, somar, acrescentar. O símbolo dessa operação é o "+".\\

\begin{soma}
    \centering \textit{a + b = c}\\
Onde \textit{a,b} e \textit{c} são Reais.
\end{soma}
\subsubsection{Propriedades}
\begin{enumerate}
    \item \textbf{Comutatividade:} Se \textit{a} e \textit{b} são números reais, segue que,\\
    
        \begin{comu} 
            \centering \textit{a + b = b + a}\\
        \end{comu}\\
    
        Exemplo:\\
        3 + 2 = 2 + 3 = 5\\
        8 + 10 = 10 + 8 = 18\\
        
    \item \textbf{Associatividade:} Se \textit{a}, \textit{b} e \textit{c} são números reais, segue que,\\

        \begin{asso} 
            \centering \textit{a + (b + c) = (a + b) + c}\\
        \end{asso}\\

        Exemplo:\\
        5 + (1 + 4) = (5 + 1) + 4 = 10\\
        7 + (2 + 2) = (7 + 2) + 2 = 11\\

    \item \textbf{Elemento neutro:} O \textit{0} é conhecido como o elemento neutro da adição. Ou seja, se \textit{a} é um número real \textit{a + 0 = a}\\
    
        Exemplo:\\
        5 + 0 = 5\\

    \item \textbf{Simétrico:} Se \textit{a} é um número real, pode-se dizer que \textit{-a} é o oposto ou simétrico de \textit{a}, da forma que\\
    
        \begin{sime} 
            \centering \textit{a + (-a) = 0}\\
            \centering \textit{a = a}\\
        \end{sime}\\
    
        Exemplo:\\
        3 + (-3) = 0\\
        8 + (-8) = 0 \\
\end{enumerate}
\subsection{Subtração} 
É o proesso oposto da adição, ou seja, retirar, remover, subtrair. Todas as propriedades da adição valem aqui. É da forma:\\

\begin{sub}
    \centering \textit{a - b = c}\\
    Onde \textit{a, b} e \textit{c} são Reais.
\end{sub}



\end{document}
